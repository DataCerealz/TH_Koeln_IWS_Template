\begin{abstract}
    Platz für das deutsche Abstract...
  \end{abstract}
  
  \renewcommand{\abstractname}{Abstract}
  \begin{abstract}
    Platz für das englische Abstract...
  \end{abstract}
  %<MERKKASTEN> (für die eigene Verwendung bitte entfernen
    \vspace{1cm}
  \begin{tcolorbox}[title={Das Abstract}]
Bei einem Abstract handelt es sich um eine Art \textit{Zusammenfassung} Ihrer Arbeit. Diese kann in deutscher und/oder englischer Sprache verfasst werden. Mithilfe des Abstracts kann der Leser sich zügig orientieren, in wie fern Ihre Arbeit für ihn Relevanz besitzt.\\                                                                      Sprechen Sie unbedingt mit Ihrer Betreuerin/Ihrem Betreuer, ob Sie für Ihre Arbeit ein Abstract benötigen.\\
Ein Abstract beinhaltet folgende Aspekte \footnote{ Vgl. \cite{SW11}, S. 249}:
\begin{itemize}
 \item Ziel der Arbeit
 \item Fragestellung der Arbeit
  \item Herangezogener, theoretischer Ansatz ("Quellen")
  \item \textit{Optional:} Methodik
\end{itemize}
  \end{tcolorbox}
  %</MERKKASTEN>

%<MERKKASTEN> (für die eigene Verwendung bitte entfernen
  \vspace{1cm}
  \begin{tcolorbox}[title={Hinweise zu dieser Dokumentvorlage}]
  \begin{itemize}
   \item Es handelt sich hierbei um eine Beispiel-Vorlage für wissenschaftliche Ausarbeitungen.
Über die konkrete, formale Ausgestaltung Ihrer wissenschaftlichen Arbeit sprechen Sie unbedingt mit Ihre/m Betreuer/in.
  \item Unabhängig, ob Sie beispielsweise eine Bachelor-, Master- oder Hausarbeit schreiben müssen. Diese Vorlage kann als eine gute Basis für Ihre Arbeit dienen. Passen Sie einfach die Vorlage Ihren Anforderungen entsprechend an.
  \end{itemize}
  \end{tcolorbox}
  %</MERKKASTEN>  