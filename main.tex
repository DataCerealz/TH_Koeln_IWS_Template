%%%%%%%%%%%%%%%%%%%%%%%%%%%%%%%%%%%%%%%%%%%%%%%%%%%%%%%%%%%%%%%%%%%%%%%%%%%%%%%
%% LaTeX-Vorlage für Abschlussarbeiten                                       %%
%% (TH Köln -Campus Gummersbach, Fak. 10)                                    %%
%%%%%%%%%%%%%%%%%%%%%%%%%%%%%%%%%%%%%%%%%%%%%%%%%%%%%%%%%%%%%%%%%%%%%%%%%%%%%%%

%%%%%%%%%%%%%%%%%%%%%%%%%%%%%%%%%%%%%%%%%%%%%
%% HEADER                                  %%
%%%%%%%%%%%%%%%%%%%%%%%%%%%%%%%%%%%%%%%%%%%%%
\documentclass[a4paper,12pt,oneside]{article}
% Optionen:
% - a4paper => DIN A4-Format
% - 12pt    => Schriftgröße (weitere  
%              grundlegende Fontgrößen: 10pt, 11pt)
% - oneside => Einseitiger Druck

% TODO
%Schriftgrößen: Fließtext: 12pt, Anmerkungen: 10pt, Überschriften: 14pt,
%· Zeilenabstand im Fließtext 1,5 (= genau 18pt), längere Zitate im Text (= genau 12 pt)
%und
%· Anmerkungen einfach (= genau 10pt),

%% Verwendete Pakete:
\usepackage{packages}

\addbibresource{references.bib}

\citetrackertrue


%\ifciteseen{⟨true⟩}{⟨false⟩} %if already cited anywhere before

%\ifciteibid{⟨true⟩}{⟨false⟩}
%Expands to ⟨true⟩ if the entry currently being processed is the same as the last one, and to ⟨false⟩ otherwise.

%\iffieldequals{⟨field⟩}{⟨macro⟩}{⟨true⟩}{⟨false⟩}
%Expands to ⟨true⟩ if the value of the ⟨field⟩ is equal to the definition of ⟨macro⟩, and
%to ⟨false⟩ otherwise.

%\DeclareCiteCommand{\myfootercitation}[\mkbibfootnote]
%  {
%    \usebibmacro{prenote}
%  }
%  {\ifciteibid{REPEATED.}{
%  \ifciteindex{\indexfield{indextitle}}{}%
%   \textit{\usebibmacro{author}: }
%   \printfield[citetitle]{labeltitle},
%   %\usebibmacro{year},
%   %\usebibmacro{pages},
%   \printfield[cite]{}
%   \setunit{\adddot\space}
%   %\usebibmacro{url+urldate}
%   }}
%  {\multicitedelim}
%  {\usebibmacro{postnote}}
  
%  1 Ursula Kirchhoff: Die achtziger Jahre. In: Geschichte der deutschen Kinder- und Jugendliteratur, 1990, S. 354-371, hier: S. 356.
%· Zweite und weitere Erwähnungen mit Kurztitel: ÿ 2 Kirchhoff: Die achtziger Jahre, S. %358.
%· Bezieht sich ein Literaturnachweis auf ein Werk, das in der unmittelbar vorhergehenden %Anmerkung genannt ist, kann der Nachweis lauten:
%ÿ 3 ebd. [ebenda]
%· Im Literaturverzeichnis wird der vollständige Titel genannt:
%ÿ Kirchhoff, Ursula: Die achtziger Jahre. In: Geschichte der deutschen Kinder- und
%Jugendliteratur ,
%· hrsg. von Reiner Wild, Stuttgart: Metzler 1990, S. 354-371.
  
  
 
%\DeclareCiteCommand{\citefirst}[\mkbibfootnote]

%\DeclareCiteCommand{\citesecond}[\mkbibfootnote]

%\DeclareCiteCommand{\citerepeat}[\mkbibfootnote]

% INFO: Zeilenabstand setzen:
%
% Befehle:
% - \singlespacing  => 1-zeilig (Standard)
% - \onehalfspacing => 1,5-zeilig
% - \doublespacing  => 2-zeilig
\onehalfspacing % Zeilenabstand auf 1,5-zeilig setzen

%%%%%%%%%%%%%%%%%%%%%%%%%%%%%%%%%%%%%%%%%%%%%
%% DOKUMENT                                %%
%%%%%%%%%%%%%%%%%%%%%%%%%%%%%%%%%%%%%%%%%%%%%
\begin{document}
  % Unbeschriftetes Vorblatt (Leere Seite)
  \pagestyle{empty} % Seite ohne Kopf- und Fußzeilen
  \newpage % Neue Seite
  
  % Deckblatt
  \pagestyle{empty}
  \begin{titlepage}
    \includegraphics[scale=1.00]{sources/logo_TH-Koeln_CMYK_22pt.eps}\\
    \begin{center}
      \Large
      Technische Hochschule Köln\\
      Fakultät für Kommunikations- und Informationswissenschaften\\
      \vspace{0.5cm}
      \hrule\par\rule{0pt}{2cm} % Horizontale Trennlinie  mit 2 cm Abtand nach unten erzeugen
      %\LARGE
      %\textsc{B A C H E L O R A R B E I T}\\
      %\vspace{1cm} % Vertikaler Abstand von 1cm erzeugen
      \huge
      Titel der Arbeit\\
      \Large
      Ggf. Untertitel\\
      \vspace{1.5cm}
      \large
      Vorgelegt an der TH Köln\\
      Campus Südstadt\\
      im Studiengang\\
      Data and Information Science\\ 
      \vspace{1.0cm}
      ausgearbeitet von:\\
      \textsc{Paul Nebatz}\\
      (Matrikelnummer: 11140591)\\
      \vspace{1.5cm}
      %\begin{tabular}{ll} % Einfache Tabelle ohne Rahmen, mit 2 Spalten erzeugen
      %    \textbf{Erster Prüfer:} & <Name des 1. Prüfers> \\
      %    \textbf{Zweiter Prüfer:} & <Name des 2. Prüfers> \\
      %\end{tabular}
      %\vspace{1.5cm}
      Köln, im <Monat der Abgabe>
    \end{center}    
  \end{titlepage}
  \newpage
  
  % Abstract (ACHTUNG: Abweichung zur Reihenfolge im Merkblatt!)
  \begin{abstract}
    Platz für das deutsche Abstract...
  \end{abstract}
  
  \renewcommand{\abstractname}{Abstract}
  \begin{abstract}
    Platz für das englische Abstract...
  \end{abstract}
  %<MERKKASTEN> (für die eigene Verwendung bitte entfernen
    \vspace{1cm}
  \begin{tcolorbox}[title={Das Abstract}]
Bei einem Abstract handelt es sich um eine Art \textit{Zusammenfassung} Ihrer Arbeit. Diese kann in deutscher und/oder englischer Sprache verfasst werden. Mithilfe des Abstracts kann der Leser sich zügig orientieren, in wie fern Ihre Arbeit für ihn Relevanz besitzt.\\                                                                      Sprechen Sie unbedingt mit Ihrer Betreuerin/Ihrem Betreuer, ob Sie für Ihre Arbeit ein Abstract benötigen.\\
Ein Abstract beinhaltet folgende Aspekte \footnote{ Vgl. \cite{SW11}, S. 249}:
\begin{itemize}
 \item Ziel der Arbeit
 \item Fragestellung der Arbeit
  \item Herangezogener, theoretischer Ansatz ("Quellen")
  \item \textit{Optional:} Methodik
\end{itemize}
  \end{tcolorbox}
  %</MERKKASTEN>

%<MERKKASTEN> (für die eigene Verwendung bitte entfernen
  \vspace{1cm}
  \begin{tcolorbox}[title={Hinweise zu dieser Dokumentvorlage}]
  \begin{itemize}
   \item Es handelt sich hierbei um eine Beispiel-Vorlage für wissenschaftliche Ausarbeitungen.
Über die konkrete, formale Ausgestaltung Ihrer wissenschaftlichen Arbeit sprechen Sie unbedingt mit Ihre/m Betreuer/in.
  \item Unabhängig, ob Sie beispielsweise eine Bachelor-, Master- oder Hausarbeit schreiben müssen. Diese Vorlage kann als eine gute Basis für Ihre Arbeit dienen. Passen Sie einfach die Vorlage Ihren Anforderungen entsprechend an.
  \end{itemize}
  \end{tcolorbox}
  %</MERKKASTEN>  
  
  \newpage
  
  % Inhaltsverzeichnis
  \tableofcontents
  % Wenn ein Punkt untergliedert wird, muss es mindestens zwei Unterpunkte geben
  % Die Gliederung sollte möglichst nicht tiefer als drei- bis vierstufig sein
  
  \newpage
  \pagestyle{fancy} % Kopf- und Fußzeilen aktivieren (=> Paket "fancyhdr")
 
  % Abbildungsverzeichnis  
  % INFO: Abbildung einbinden (Beispiel):
  %  \begin{figure}[h!]
  %    \centering
  %    \includegraphics[scale=1.00]{Pfad zum Bild}\\
  %    \caption{Bildunterschrift} 
  %    \label{Marke zum Referenzieren auf die Abbildung}
  %  \end{figure}
  
  
  \newpage
  
  % Hauptteil des Dokuments
  \newpage
  % INFO: Querverweise auf Gliederungselemente, Abbildungen 
  %       & Tabellen setzen:
  %
  % Voraussetzung: Gesetzte Referenzmarke mit dem Befehl: \label{marke}
  % 
  % Referenzierung erfolgt dann mittels dem Befehl:
  % \ref{marke}
  
  \section{Einleitung}\label{kap_einleitung}  
   
   TEXT FOLGT...
   
    %<MERKKASTEN> (für die eigene Verwendung bitte entfernen
    \vspace{1cm}
 \begin{tcolorbox}[title={Die Einleitung}]
Die Einleitung umfasst folgende Elemente\footnote{Vgl. u.a. \cite{BBoJ}, S. 5-6}:
\begin{itemize}
 \item Einführung in das Thema (Motivation, zentrale Begriffe etc.)
 \item Hinführung zu den Ergebnissen
 \item Ggf. Angabe des Schwerpunktes
 \item Ggf. Einschränkungen darlegen
 \item Problemstellung
 \item Zielstellung der Arbeit
 \item Fragestellung der Arbeit
 \item Übersicht über die Kapitel geben: 
\begin{quotation}
Eine Einleitung muss auch durch die Arbeit führen. Sie muss dem Leser helfen, sich in der Arbeit und ihrer Struktur zu Recht zu finden. Für jedes Kapitel sollte eine ganz kurze Inhaltsangabe gemacht werden und ggf. motiviert werden, warum es geschrieben wurde. Oft denkt sich ein Autor etwas bei der Struktur seiner Arbeit, auch solche Beweggründe sind dem Leser zu erklären\footnote{\cite{BBoJ}, S. 6}:. 
\end{quotation}
\end{itemize}
  \end{tcolorbox}
  %</MERKKASTEN>
   
  \newpage  
  \section{Grundlagen}\label{kap_grundlagen}  
    TEXT FOLGT...
   
    \subsection{Unterabschnitt von Grundlagen}\label{subsec_UabsGrundl}
     
    Hierbei \autocite{einstein} handelt es sich um ein Beispiel-Kapitel. Es ist zu empfehlen, dass Sie Kapitel und auch Abschnitte immer mit\autocite{SW11} einer kurzen Einleitung beginnen. In dieser beschreiben Sie kurz\autocite[125]{einstein}, was den Leser in diesem Kapitel/Abschnitt erwartet. Bei\autocite{SW11} einem Kapitel mit Abschnitten nehmen Sie auch inhaltlichen Bezug auf die enthaltenen Abschnitte (inklusive\autocite{SW11} Referenzierung auf die Abschnittsnummerierung).
      
    Bei Verwendung von Tabellen und auch Abbildungen beachten Sie bitte, dass diese immer Unter-/Überschriften enthalten (inklusive einer Nummer). Im Textfluss erklären/beschreiben Sie die Abbildung bzw. die Tabelle und nehmen Bezug über einen Verweis auf die Nummer.
  
  
  \newpage 
  \section{Zusammenfassung und Ausblick}\label{kap_zusammfAusbl}  
   %\input{}
   TEXT FOLGT...
   
     %<MERKKASTEN> (für die eigene Verwendung bitte entfernen
    \vspace{1cm}
 \begin{tcolorbox}[title={Inhalte der \textit{Zusammenfassung und Ausblick}}]
Das Kapitel \textit{Zusammenfassung und Ausblick} enthält folgende formale Aspekte\footnote{Vgl. \cite{BBoJ},S. 6}:
\begin{itemize}
\item Kapitelweise \footnote{ \cite{einstein} } Kurzdarstellung der Inhalte (inklusive Referenzierung auf die Kapitelnummerierung) => Nach dem Motto: \textit{Was wurde wo beschrieben?}
\item Kurzdarstellung \textit{Problem – Lösungsweg – Ergebnisse}
\item Rückkopplung auf die Einleitung: Wurde die Zielstellung der Arbeit und die Fragestellung zufriedenstellend beantwortet?
\item Kritische Bewertung (sofern nicht bereits im Hauptteil geschehen)
\item Offene Probleme
\item Richtung der zukünftigen/möglichen Arbeiten
\item Erläuterung, warum welche Aspekte in der Arbeit nicht erläutert wurden
\end{itemize}
  \end{tcolorbox}
  %</MERKKASTEN>   
   
  % Literaturverzeichnis
   % INFO: Referenzieren auf das Literaturverzeichnis:
   %
   % Befehl: \cite{refmarke}
   % 
   % "refmarke" ist die Angabe in den geschweiften Klammern bei 
   % \bibitem[]{refmarke}. 
   \newpage
    \pagestyle{empty}
   \section{Quellenverzeichnis}
  
  % INFO: Biblatex -Ausgabe des  
  % Literaturverzeichnisses (Beispiele):   
  \printbibliography % - => Ausgabe ALLER 
  %   Einträge
  % - \printbibliography[nottype=online]
  %   => Ausgabe der Einträge, bis auf die
  %      "Online"-Einträge
  % - \printbibliography[type=online]     
  %   => Ausgabe nur der "Online"-Einträge  
   %\printbibliography
   
   
   % Abbildungsverzeichnis
   \newpage
    \section*{Abbildungsverzeichnis}
  \addcontentsline{toc}{section}{Abbildungsverzeichnis} % Manuellen Eintrag im Inhaltsverzeichnis erzeugen
  \renewcommand{\listfigurename}{} % Name des Abbildungsverzeichnisses ändern
  \listoffigures
  
  % Tabellenverzeichnis
  \newpage
  \addcontentsline{toc}{section}{Tabellenverzeichnis}
  \listoftables

  % Anhang
  \newpage
  \pagestyle{fancy}
  \setcounter{section}{0} % Nummerierung der Gliederungsebene "section" auf 0 setzen
  \renewcommand*\thesection{\Alph{section}} % Nummerierungsart für die Gliederungsebene "section" 
  % auf Großbuchstaben setzen
  \section{Anhang}\label{anhang}
    \subsection{Unterabschnitt von Anhang}\label{subsec_UabsAnhang}
    TEXT FOLGT...
  
  \newpage
  \pagestyle{empty}
  
  % Erklärung zur eigenständigen Arbeit
 \include{selbststaendige_arbeit}
  
%\newpage
% Unbeschriftetes Abschlussblatt (Leere Seite)
%\input{leereSeite}
    
\end{document}

